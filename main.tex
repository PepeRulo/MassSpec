\documentclass{article}
\usepackage[utf8]{inputenc}
\usepackage{setspace}
\doublespacing
\usepackage{enumitem}

\title{Mass Spectroscopy}
\author{Jose Raul Pastrana }
\date{October 2015}

\begin{document}

\maketitle

\section{Introduction}

Spectroscopy is the branch of science that deals with measurements of electromagnetic radiation. Mass spectrometry is an analytical technique that involves analyzing fragmented particles' patterns to reconstruct them and find out their chemical compositions, principally by analyzing their mass to charge ratio. Originally, the technique used to analyze the fragments required them to be recorded on plate through electromagnetic radiation, thus the name $mass \ spectroscopy$. However, nowadays, the equipment used to execute the task doesn't employ electromagnetism, making it an archaic term. Now it is, technically, referred to as $mass \ spectrometry$.

\section{Research}

Mass spectrometry is carried out by using a mass spectrometer. This instrument can measure the masses and relative concentrations of atoms and molecules by a process of ionizing and accelerating particles to later select (deflect) them. Selected single velocity particles go into the instrument's generated magnetic field in a circular path. "This is because the ions are deflected by the magnetic field according to their masses; the lighter they are, the more they are deflected" (Clark). The machine can measure the radius of the particles' followed path through detectors and, comparing the position of the impacts on them, identify the mass of the particle. The position of the impact of the particle in the detector is then a function of the particle's mass. 

Deflection can be deconstructed by the following analysis. When charged particles enter the magnetic field their direction is perpendicular to the field, causing the circular path. As the magnetic force is perpendicular to the velocity, the force that occurs is centripetal. This can be illustrated in the following way:
$$F_{net}$$
$$F_{B} = F_{centripetal}$$
$$F = qv \times B = \frac{mv^2}{r}$$

To find the path's radius:
$$qv \times B = \frac{mv^2}{r}$$
$$r=\frac{mv^2}{qv \times B}$$
$$r=\frac{mv}{qB}$$

The obtained equation gives the mass to charge ratio $\frac{m}{q}$, a physical quantity –property quantified by measurement– relevant for its proof that "two particles with the same ratio move in the same path in a vacuum when subjected to the same electric and magnetic fields". 

\newpage

The previous equation could be modified to exemplify the ratio (where $v$ is known, $B$ is set and $r$ is being measured):
$$r=\frac{mv}{qB}$$
$$\frac{m}{q}=\frac{rB}{v}$$


\section{References}

\begin{enumerate}[leftmargin=!,labelindent=5pt,itemindent=-15pt]

\item Clark, Jim. "The mass spectrometer." Chemguide. Chemguide, Feb. 2015. Web. 26 Oct. 2015.
\item "Mass Spectrometer." HyperPhysics. Georgia State University, n.d. Web. 26 Oct. 2015.
\item "Mass Spectrometry." (n.d.): (Chemistry) University of Minnesota. University of Minnesota. Web. 26 Oct. 2015.
\item Soup. "Is There a Difference between Mass Spectroscopy and Mass Spectrometry?" Yahoo! Answers. Yahoo!, 2009. Web. 26 Oct. 2015.
\item Wikipedia contributors. "Mass spectrometry." Wikipedia, The Free Encyclopedia. Wikipedia, The Free Encyclopedia, 18 Oct. 2015. Web. 26 Oct. 2015.
\item Wikipedia contributors. "Mass-to-charge ratio." Wikipedia, The Free Encyclopedia. Wikipedia, The Free Encyclopedia, 26 Sep. 2015. Web. 26 Oct. 2015.

\end{enumerate}



\end{document}
